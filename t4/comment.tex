\documentclass[11pt]{article}

\usepackage[T2A]{fontenc}
\usepackage[utf8]{inputenc}
\usepackage[russian]{babel}

\usepackage{hyphenat}
\hyphenation{ма-те-ма-ти-ка вос-ста-нав-ли-вать}

\usepackage[a4paper,margin=2cm]{geometry}

\usepackage{graphicx}
\usepackage[intlimits]{amsmath}
\usepackage{amssymb}
\usepackage{amsmath}
\usepackage{subcaption}
\usepackage{wrapfig}
\usepackage{float}
\usepackage{fancyhdr}
\usepackage{mathtools}
\usepackage{tensor}
\usepackage{pgf}
\usepackage[parfill]{parskip}
\usepackage{array}
\usepackage[utf8]{inputenc}\DeclareUnicodeCharacter{2212}{-}

\newcommand{\rot}[1]{[\nabla, \mathbf{#1}]}
\newcommand{\di}[1]{(\nabla, \mathbf{#1})}
\newcommand{\ve}[1]{\mathbf{#1}}
\newcommand{\re}[1]{(\ref{#1})}


\begin{document}
 Рассмотрим краевую задачу в общем виде:
 \[
\begin{dcases} 
   y'' + p(x)y' + q(x)y = f(x) \\
   \alpha_1 y(a) + \beta_1 y'(a) = \gamma1 \\
   \alpha_2 y(b) + \beta_2 y'(b) = \gamma2 \\
\end{dcases}
\]
Введя сетку $hN = b - a$ и заменив производные разностными соотношениями, получим:
\[
 \begin{dcases}
  \alpha_1 y_0 + \beta_1 \frac{y_1 - y_0}{h} = \gamma1 \\
  \frac{y_{n+1} - 2y_n + y_{n-1}}{h^2} + p_n\frac{y_{n+1} - y_{n-1}}{2h} + q_ny_n = f_n \\
  \alpha_2 y_N + \beta_2 \frac{y_n - y_{N-1}}{h} = \gamma2 \\
 \end{dcases}
\]
Тогда получим систему, задающую тридиагональную матрицу, решаемую методом прогонки:
\[
 \begin{dcases}
  y_0\left(\alpha_1 - \frac{\beta_1}{h}\right) + \frac{\beta1}{h}y_1 = \gamma1 \\
  y_{n+1} \left(\frac{1}{h^2} + \frac{p_n}{2h}\right) + y_n\left(-\frac{2}{h^2} + q_n\right) + y_{n-1}\left(\frac{1}{h^2} - \frac{p_n}{2h}\right) = f_n\\
  y_{N}\left(\alpha_2 + \frac{\beta_2}{h}\right) - \frac{\beta2}{h}y_{N-1} = \gamma2 \\
 \end{dcases}
\]
\end{document}
